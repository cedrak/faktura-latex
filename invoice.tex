\documentclass[12pt]{article}
\usepackage{polski}
\usepackage[utf8]{inputenc}

\setlength{\marginparwidth}{0pt}
\setlength{\parindent}{0pt}
\addtolength{\hoffset}{-50pt}
\pagestyle{empty}

% Dane do faktury
\newcommand{\datawystawienia}{31.07.2019}
\newcommand{\terminplatnosci}{14.08.2019}
\newcommand{\nrfaktury}{0/2019} 
\newcommand{\netto}{0.00} 
\newcommand{\vat}{0.00}
\newcommand{\brutto}{0.00}
\newcommand{\slownie}{zero 0/100}
\newcommand{\product}{Adwokat diabła}
% Koniec danych do faktury

\begin{document}

\begin{tabular}{p{0.8\textwidth} l l}
    \textit{Sprzedawca:} & Data wystawienia: & \datawystawienia \\
    Moja firma & & \\
    ul. Słoneczna 42/24 & & \\
    00-000 Kolonia & & \\
    NIP: 0000000000 & & \\ 
                   & & \\
    Bank: Mój Bank & & \\
    Konto: 00 0000 0000 0000 0000 0000 0000 & & \\
\end{tabular}
\ \\ \ \\
\centerline{\hspace{50pt}\LARGE{Faktura VAT nr \nrfaktury}}\\
% \centerline{\hspace{50pt}(oryginał / kopia)}\\
\ \\
\textit{Nabywca:} \\
\ \\
Firma Nabywająca Sp. z o.o.\\
ul. Nabywców 42, 00-000 Nabytów\\
NIP: 0000000000\\

\begin{tabular}{r p{6cm} l r p{2cm} l r p{2cm}}
    \hline 
    Lp & Nazwa usługi & j.m. & Ilość & Wartość netto
    & VAT & Podatek & Wartość z podatkiem \\ 
    \hline
    % sprzedane usługi 
    % Numer pozycji & Nazwa & Jednostka & Ilość & Wartość netto
    % & VAT & Podatek & Wartość z podatkiem
    1 & \product & szt & 1 & \netto & 23 \% & \vat & \brutto \\
    \hline
    & & & Razem: & \netto & & \vat & \brutto \\
    \\
\end{tabular}

\ \\
Słownie: \slownie\\
\ \\
Forma płatności: PRZELEW \\
Termin płatności: \terminplatnosci \\


\ \\ \ \\ \ \\
..............................\hspace{220pt}..............................\\
\ \ \ Osoba odbierająca\hspace{225pt}\mbox{Osoba wystawiająca}

\end{document}
